% Chapter 2

\chapter{Background}
\label{ch:background}
\lhead{Chapter~\ref{ch:background}. \emph{Background}}

%\section{Underwater Localization}

\section{AUVs and Localization}

The words \textit{Positioning} and \textit{localization} are often used interchangeably in literature. Positioning, although similar to localization, also has the connection of placing the vehicle in a particular place. In this thesis, we will use the word \textit{localization} to mean the process of locating the vehicle, with or without the reference of a map.

The sensors for underwater localization are categorized into two groups. The first group of sensors do not require external infrastructure for localization. For example, DVL measures vehicle speed relative to the seabed. A microelectromechanical system (MEMS), usually called as compass, measures the vehicle's 3-axis orientation. An inertial measurement unit (IMU) provides the vehicle's acceleration and angular rate with respect to the vehicle's body-frame. Other examples include depth sensor, altimeter, sidescan sonar, and forward looking sonar (FLS). Localization by these sensors usually observe the natural environment as external references and can be performed in any environment. However these sensors often do not give full information about vehicle's location. The most common localization method of localization with these sensors is dead reckoning (DR). It calculates the vehicle's position by integration of velocity over time. Accurate DR systems tend to be expensive as high-grade sensors are needed. Even though, any small bias or offset leads to an unbounded increase in error over time while AUV remains submerged. The second group are sensors which require additional infrastructure setup. Acoustic positioning systems measure vehicle's range and/or angle to the beacons and therefore provide direct estimates of the position relative to beacons. The beacons can be fixed to sea bottom, or the surface vehicles, or even other AUVs. These sensors give good localization reference but often incurs high cost in deployment and operation.

In a typical localization problem, the state to be estimated usually consists of the vehicle's position, depth, orientation (heading, pitch and roll) and velocity. During a mission, the autonomous vehicle estimates its own state and at the same time uses the estimated state for further actions. Usually the depth of an AUV is specified in a mission and measured by the depth sensor directly. Therefore in this thesis we only focus on the vehicle's position in the 2-dimensional space, that is, the northing-easting position.

Let ${\bf x}_k$ be the state vector to be estimated at time step $k$ where ${\bf x}_k\in\mathbb{R}^n$, and let the estimate of ${\bf x}_k$ be ${\bf \hat x}_k$ (In some illustration in this thesis, we also use ${\bf y}_k$ to denote the estimate of state ${\bf x}_k$).  The general system evolution is modeled as
\begin{equation}
\label{eq:propagation}
{\bf x}_{k+1}=f({\bf x}_k,{\bf a}_k)+{\bf\omega}_k
\end{equation}
where ${\bf a}_k$ is the action that directs vehicle movement for the next step. The behavior of the system is observed through measurement ${\bf z}_k\in\mathbb{R}^m$. The measurement model is
\begin{equation}
\label{eq:measurement}
{\bf z}_{k+1}=h({\bf x}_{k+1})+{\bf\nu}_{k+1}.
\end{equation}
The process noise ${\bf\omega}_k$ and measurement noise ${\bf\nu}_k$ are Gaussian white-noise sequences and are mutually independent. Their error covariances are ${\bf Q}_k$ and ${\bf R}_{k+1}$ respectively.

Typically the vehicle carries a belief as to where it might be, and maintains the localization as a probability distribution over the space of all such hypotheses. Knowledge of the probability density function (PDF) of the state conditioned on all available measurement data ${\bf Z}_{1:k}\triangleq\{{\bf z}_1,{\bf z}_2,...,{\bf z}_k\}$ provides the most complete possible description of the state. Given \textit{a posteriori} density function $p({\bf x}_k|{\bf Z}_{1:k})$, estimate ${\bf \hat x}_k$ of the state is obtained from some performance criteria, for example, maximum \textit{a posteriori} (MAP). Bayesian estimation \cite{Aoki1967} recursively determines the a posteriori density $p({\bf x}_k|{\bf Z}_{1:k})$ as
\begin{align}
\begin{split}
p({\bf x}_k|{\bf Z}_{1:k})&=\frac{p({\bf z}_{k}|{\bf x}_k)p({\bf x}_k|{\bf Z}_{1:k-1})}{p({\bf z}_k|{\bf Z}_{1:k})} \\
&=\alpha p({\bf z}_{k}|{\bf x}_k)p({\bf x}_k|{\bf Z}_{1:k-1}) \\
\end{split}
\end{align}
where
\begin{align}
\begin{split}
\frac{1}{\alpha}&\triangleq p({\bf z}_k|{\bf Z}_{1:k}) \\
&=\int p({\bf z}_{k}|{\bf x}_k)p({\bf x}_k|{\bf Z}_{1:k-1})\mathrm{d}{\bf x}_{k}.
\end{split}
\end{align} % wiki recursive bayesian estimation
The simplification is based on Markov assumption \cite{Alspach1972} which states that if one knows the vehicle location ${\bf x}_k$, future measurements are independent of the past ones, that is
\begin{equation}
    p({\bf z}_k|{\bf x}_k,{\bf Z}_{1:k-1})=p({\bf z}_k|{\bf x}_k).
\end{equation}
The density $p({\bf z}_{k}|{\bf x}_k)$ is the \textit{a priori} distribution describing the sensor model. The state transition model is described by the prediction as
\begin{equation}
p({\bf x}_k|{\bf Z}_{1:k-1})=\int p({\bf x}_{k}|{\bf x}_{k-1})p({\bf x}_{k-1}|{\bf Z}_{1:k-1})d{\bf x}_{k-1}.   
\end{equation}

When both evolution and measurement models are linear with additive Gaussian noises, and the \textit{a priori} distribution is Gaussian, Kalman filter (KF) \cite{welch1995} can be used to derive the closed-form solution by simply using the estimated mean ${\bf\hat x}$ and estimated covariance ${\bf P}$ for this system. The parametric description of the distribution is efficient in integrating the motion and updating the estimates, and also easy to evaluate the localization performance. However it is only suitable for linear systems with Gaussian noises. Extended Kalman filter (EKF) \cite{Julier2004} is the nonlinear version of the KF which linearizes the nonlinear evolution and/or nonlinear measurements. Only unimodal distributions can be modeled by KF and its extensions. Nonparametric filters use numerical approaches to describe the PDF and are particularly suitable for nonlinear and non-Gaussian system estimation since their probability function evolves to better fit the data. The most well-known nonparametric method is the Particle filter (PF) \cite{PFtutorial}.

\section{Beacon-Based Localization}

In beacon-based localization, beacons are placed in the area where AUVs navigate. For example, long baseline acoustic positioning system (LBL) places the transponders on the seafloor, while short baseline system (SBL) and ultra short baseline system (USBL) have the transponders mounted on a ship that follows the vehicle. These methods differ in the distances between the transponders and the distance from each transponder to the vehicle \cite{Wolbrecht2014}. The GIB system uses buoys on the surface \cite{Alcocer2006}. When fixed beacons are used \cite{AlexPhillips2018B}, AUVs are limited in their exploration area. The deployment and setup of the infrastructure are tedious and expensive. Some systems assist the localization with moving beacons. The moving beacons can be mounted on vessels \cite{Curcio2005, Webster2013,AlexPhillips2018} or even other AUVs \cite{Baccou2001,Deshpande2007,Gao2010}. Surface vessels often encounter danger of collision with other traffic on the surface.
Beacon AUVs are usually assumed to have precise positioning. Generally the beacon vehicle operates at the surface and has access to GPS, or is equipped with high-accuracy sensors which enables it to estimate its own position with minimum errors. A setup, which consists of beacon AUVs with precise positioning, can form a cooperative team of heterogeneous AUVs. 

\section{Cooperative Localization}

The idea of cooperative localization with beacon AUV is not new.
Authors in \cite{Gao2010,William2014Journal} presented a single-beacon vehicle providing range-only measurement to support the localization of other AUVs. The supported AUVs are survey AUVs equipped with sensors for mission purposes. For example, LEDIF sensor \cite{NgLEDIF} was installed on STARFISH AUVs \cite{STARFISH} for chemical sensing. With these sensing units, survey AUVs are often equipped with poor navigation sensors. The distance between the survey AUV and the beacon vehicle is measured to impose a limit on the position drift of the survey AUV. This approach has been explored by several works which use observability analysis \cite{Antonelli2010,Song1999}, and position determination algorithms \cite{Gadre2004,Alleyne2000}. However, these works pay little attention to the path planning of the beacon vehicle. For example,  \cite{Alleyne2000} assumed a circular path for the beacon vehicle, \cite{Fallon2010} used a zigzag path during experiments and \cite{Webster2012} adopted a diamond shaped path.

It is acknowledged that the relative motion between the beacon vehicle and the survey AUVs is key to having single beacon range-only navigation perform well. The path of the beacon vehicle should be planned in such a way that it improves the position estimate of the survey AUVs. In \cite{chitre2010}, the path planning of the beacon vehicle aimed at minimizing the accumulated localization errors in the supported survey AUVs. In \cite{Bahr2009beacon}, the optimal beacon point targeted at minimizing the position uncertainty of the survey AUV, but it was determined by brute-force searching approach. Later \cite{William2014Journal} proposed the cooperative path planning algorithm using dynamic programming and Markov decision process formulations.

In the waters of Singapore, heavy maritime traffic makes it dangerous and inconvenient for AUVs to surface for GPS fix. Consequently, even beacon AUVs with high-grade sensors suffer from position drift. We look at this problem and propose a solution whereby no single AUV functions as a beacon possessing accurate position information in the team. In terms localization capability, the team of AUVs is a homogeneous team, whereas a heterogeneous team includes a beacon AUV with precise position. The relative position or range between these AUVs can be considered as a relative geometry constraint in
localization. This cooperative localization is easier to be understood if we treat the group of AUVs as one identity. The individual AUVs are the multiple ``limbs" while the distances between AUVs are the multiple virtual ``joints". Each limb moves on its own path, and from time to time, when the distance between two AUVs is measured, the position estimates of all the AUVs are
adjusted accordingly. Cooperative multi-AUV localization has the potential to outperform single-AUV localization, by taking advantage of data sharing among the team members. However, it should be noted that many factors affect the communication and subsequently the cooperation performance. Water temperature, salinity, underwater noise, Doppler phase shift, reflection and scattering of seabed and sea surface are relevant contributing factors to be considered. Underwater communications have limited bandwidth and are less reliable compared with terrestrial communication links. Full
communication with every member of the team at any one time of instant is often not practical.

In such a case, decentralization in processing and navigation becomes necessary. One example of decentralized processing comes from the self-organized behavior, namely \textit{Swarm Intelligence (SI)}. SI system consists of a population of simple agents, which interact locally with one another following simple rules. There is no centralized unit or individual who knows the full details or dictates how each member should behave. Interactions between agents lead to an ordered or ``intelligent'' global behavior. The phenomenon is often referred to as `emergence'.  The inspiration often comes from nature, especially biological system behavior like ant colonies, bird flocking, animal herding and fish schooling. The application of
SI to robotics is called `swarm robotics', a method of coordinating large numbers of simple robots, which interact with one another to give rise to a desired collective behavior \cite{SI1,SI2,SI3}. In ocean studies, the use of swarm AUVs for data gathering purposes has emerged as an attractive and alternative solution to the tedious and manual process of deploying sensor probes, for example beacons installed on surface vessels. Swarm AUVs are able to gather more data than the traditional approach, operating at lower overall cost, and can be deployed to function in harsh environment. It is also robust to individual failure, compared with the single-AUV system. Swarm AUVs
coordinate their behavior in a distributed way to achieve a particular goal, such as resource sharing, synchronized motion \cite{Sam2005}, localization \cite{liu2014}, a specific swarm pattern or coverage \cite{Spires1998}, etc.

The concept of SI can be visualized in the tutorial about self-propelled particles (SPP) in \cite{Sam2005}. In this tutorial, each particle has its own randomness in movement and follows the average heading when it meets other particles within its vicinity. It introduces order parameters and other critical values to visualize how the order or phase change with respect to the variation of randomness, number of particles, vicinity range etc. In SI, the communication and coordination among members is usually minimal. The observation on other members also comes with more randomness.

\section{Dealing with Unknown Correlation}
\label{sec:unknowncorrelation}
In this thesis, we focus on a small team of AUVs. These AUVs locally estimate their own positions, and gather observations to update the position estimates. The distributed processing architecture has many advantages over centralized architecture. It is reliable in the sense that the loss of any individual AUV or links does not necessarily prevent the rest of the team from completing the mission. It is flexible in the sense that AUVs can be added or deleted from the team by making only local changes. 

The most challenge arising from distributed processing in a cooperative team is the effect of redundant information \cite{GRIME1994849}. To be specific, the information from different vehicles cannot be combined straightforward unless they are independent or have a known degree of correlation. Many years ago authors in~\cite{Bar-Shalom1986} recognized that local estimates have correlated errors. If between local estimates the correlation is \textit{na\"ively} assumed to be zero, estimation overconfidence comes with the fused estimate, and may lead to filter divergence \cite{Julier1997}. Therefore dealing with unknown correlation in the cooperative localization becomes important. With smaller number of AUVs in a group, common information flowing around the team is more influential. This is because the common information easily cycles with less independent information input from other AUVs.

A simple parametric localization filter like Kalman filter \cite{kalman1960new} and its extensions \cite{kalman1960new,KFbook,Julier2004} gives estimated position ${\bf \hat x}$ and the corresponding estimated error covariance ${\bf P}$. At each time step, a vehicle predicts its own position and updates the estimate if local measurement is available. They keep their local estimate ${\bf\hat x}^{(\cdot)},{\bf P}^{(\cdot)}$ where $(\cdot)$ denotes the vehicle's identity. There is a time when two vehicles (Vehicle $i$ and Vehicle $j$) in the group communicate, exchange their information (for example, the estimated position and estimated error covariance, ${\bf\hat x}^{(i)},{\bf P}^{(i)}$ and ${\bf\hat x}^{(j)},{\bf P}^{(j)}$), and update their respective position estimates. The other members in the group may not know this cooperation due to the loss of transmission packets. Therefore, to those members, the correlation between information provided by vehicles $i$ and $j$ is unknown. This is essentially a data fusion problem dealing with unknown correlation in distributed network. The estimation can be visualized as an estimation network where the nodes are the vehicles and the edges connecting the nodes denote the information flow within the network.

For one-step cooperation, we drop the time step $k$ for simplicity. Assuming both estimates are consistent, that is that $\mathbb{E}[({\bf\hat x}^{(i)}-{\bf x}^{(i)})({\bf\hat x}^{(i)}-{\bf x}^{(i)})^\top]\preceq{\bf P}^{(i)}$ and $\mathbb{E}[({\bf\hat x}^{(j)}-{\bf x}^{(j)})({\bf\hat x}^{(j)}-{\bf x}^{(j)})^\top]\preceq{\bf P}^{(j)}$, the fused estimate $({\bf \hat x}^{(f)},{\bf P}^{(f)})$ at Vehicle $i$ should satisfy the following \textit{fusion principles} \cite{Reece2005,Uhlmann2003}:
\begin{enumerate}
    \item Performance improvement: ${\bf P}^{(f)}\preceq{\bf P}^{(i)}$,
    \item Estimation consistency: $\mathbb{E}[({\bf\hat x}^{(f)}-{\bf x}^{(i)})({\bf\hat x}^{(f)}-{\bf x}^{(i)})^\top]\preceq{\bf P}^{(f)}$.
\end{enumerate}
An optimal fusion is derived in \cite{Chang1997} if correlation is known. To fuse estimates with unknown correlation, \cite{Julier2001book} proposed a Covariance Intersection (CI) method, which essentially provides an upper bound on the error covariance of the fused estimate. It leads to various approaches for determining the weight $w$ \cite{Hanebeck2001,Chen2002}. Although the estimation consistency is maintained, the estimated error covariance is no smaller than the individual error covariance at any direction. The expression of the fused estimate is derived by assuming independent errors between ${\bf\hat x}^{(i)}-{\bf x}^{(i)}$ and ${\bf\hat x}^{(j)}-{\bf x}^{(j)}$, which is contradictory to the problem formulation. Reference \cite{Sijs2010} proposed Ellipsoid Intersection (EI), which essentially picks the estimate with smaller error covariance. It fulfills the first fusion principle but the consistency is not proved. Along the same line as CI, the author in \cite{Lihao2013A} and his related work proposed Split Covariance Intersection (SCI), which uses split form to represent the dependent and independent parts in the estimate. However, in the state update with relative position estimates and states of other vehicles, only the relative measurement is separated as independent information. 

Instead of overestimating the intersection region, authors in \cite{Benaskeur2002} proposed a largest ellipsoid algorithm which leads to a tighter estimate but the fused estimate is slightly underestimated. In place of CI, \cite{Reece2005} proposed Bounded Covariance Inflation (BCInf), a mechanism for creating conservative covariance matrices for which the bounds on the cross correlations are known. The key is the book keeping messages (the coupling scalar) to interpret the correlated part and uncorrelated part. The decentralization was demonstrated by using only two-vehicle SLAM. 

\section{Bathymetry-Aided Navigation}

Bathymetry is the submerged equivalent of an above-water topographic map. It is obtained by prior survey and recorded in a geographical map. Bathymetry-based localization and navigation, is also known as terrain relative navigation (TRN) \cite{Meduna2010}, terrain based navigation (TBN) \cite{5779659,carreno2010}, terrain-aided navigation (TAN) \cite{carreno2010}, and bathymetry-aided navigation (BAN) \cite{Kalyan2013}. The key idea is to match the local bathymetry as seen by an underwater vehicle against the reference map,  and estimate the location of the vehicle on that map. Bathymetric SLAM is a well-recognized concept of navigation and map building \cite{Barkby2011,NilsBore2018} using a multibeam sonar in high-end AUVs. Multibeam sonar measures the bathymetry in a small patch of area. The measured information is rich as it consists of multiple measurements in a geographical formation. In terms of localization, rich bathymetry information from multibeam sonar gives high localization accuracy. However, multibeam sonar is costly and take up space in AUVs. As we focus on small-sized and low-cost AUVs, bathymetry measurements can be simply managed with a single echo-sounder \cite{Salavasidis2016} or altimeter. Observations from the depth sensor and altimeter lead to the observed bathymetry where the vehicle is located. Although the bathymetry is a single-point measurement, works in \cite{William2015Journal,Kalyan2013} have showed the feasibility. It is fused with the odometric estimation to get the best possible update about the vehicle position. Therefore bathymetry map indicating the water depth offers an attractive tool to reduce the localization error of the submerged AUVs, without additional cost on hardware or external infrastructure. 

In~\cite{Kalyan2013}, the authors showed a strong correlation between localization accuracy and variation of bottom topography. The conclusion was that a reasonable localization could be made with a single beam altimeter, as long as sufficient bathymetric variation was available along the AUV path. In \cite{Rodrigo2015}, bathymetry variation was characterized by the depth gradient and a path was generated to avoid regions where the terrain has small variations. In \cite{Galceran2013}, salient points (locations with more bathymetry variations) are visited when the localization uncertainty surpasses a user-provided threshold. Various terrain statistic information was listed in \cite{Peng2016}, including standard deviation, roughness, correlation coefficient and entropy. However the authors did not demonstrate these terrain information metrics and the setpoints (planned points along the path) were manually selected. The relation between the bathymetry information and localization performance has not been systematically studied. This relation first requires a quantification of localization uncertainty, which is important in relating localization performance to the bathymetry information. Generally, the undulating topology of the underwater terrain yields non-Gaussian or multi-modal distributions in localization. Traditional parametric filters such as Kalman filter are not suitable in describing such distributions or quantifying the localization performance. Secondly, the bathymetry information need to be defined properly. Information entropy map turns out to be a suitable idea matching the localization uncertainty and bathymetry information. The idea of entropy-based map has been used as an efficient method of portraying terrain data \cite{Fairbair2011}. Author in \cite{Wellmann2013} also provided an information theory framework for the analysis of spatial uncertainties. Particle filter based entropy was derived in \cite{Bpers2010} to characterize the uncertainty of a running particle filter. Up to what we know, no one has used the information theoretic measure to evaluate the navigation with bathymetric aids.