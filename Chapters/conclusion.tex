\chapter{Conclusions and Future Work} % Write in your own chapter title
\label{ch:conclusion}
\lhead{Chapter~\ref{ch:conclusion}. \emph{Conclusions and Future Work}}

A list of contributions in this thesis is summarized as:
\begin{itemize}
\item {We focus on a cooperating team of small-sized, low-cost, sensor-limited AUVs. We showed that the cooperation improves localization but also easily aggravates the performance when communication loss is higher.} 
\item {We proposed a new cooperative multi-vehicle localization algorithm using distributed extended information filter (DEIF). It is effective in recording the correlated information in light of constrained underwater communication. Simulations show that DEIF gives better performance compared with single-vehicle localization and existing cooperative localization method.}
\item {As \textit{na\"ive} filter is easy to implement in complex situation, we answer the question as to when it is safe or detrimental to ignore the correlation in cooperation.}
    \item {We formalize the concept of information entropy measure, to quantify the localization performance, and the effectiveness of bathymetry on localization. We concluded that the uniqueness of the measured depth compared with the bathymetry in the localization prior decides the localization performance.}
    \item {With the conclusion above, we proposed a path planning algorithm for navigation with bathymetric aids. The algorithm generates near-optimal paths based on bathymetry map, with good localization accuracy at the destination.}
\end{itemize}

In cooperative localization of underwater vehicles, one cannot assume that the inter-vehicle correlations are available. This is because the constrained underwater communications prevent vehicles from keeping track of all the estimates shared in the team. It may hamper the cooperation as the fused information might be false or overconfident when correlation is underestimated. We examined the problem and proposed a novel design of the distributed localization method. The method is able to record the correlated information from the most recent cooperation and transmit with small packets, providing consistent position estimates in event of packet loss.

However, ignoring correlation when fusing data seems working in some work. This dissertation studied the conditions and provided the justification where the correlation can be ignored. For multi-sensor tracking problem, the condition is when the local measurements have relatively small errors such that the local estimates are nearly independent of each other. For cooperative localization problem, besides having small local measurement errors which validates the assumption of independence, the other condition is where the local propagation is long enough such that the local estimates are nearly independent.

With the assurance of the small local measurement errors, cooperative localization with bathymetric aids can simply ignore the correlation. We explored the bathymetry-aided localization and navigation of a single vehicle. This can be extended to cooperative vehicles without the concerns of tracking inter-vehicle correlation. The effectiveness of bathymetry aids on localization is shown with an information entropy measure. It showed that the localization improvement depends on how unique the bathymetry map against a prior knowledge about the localization. With this idea, a path planning algorithm is developed for a particle filtering localization. Simulations show that the algorithm generates sub-optimal paths within a few iterations.

This research is open in applicability to other areas related to cooperative positioning and navigation. For
example, terrestrial swarm robotics on land is one very popular topic where this research can be of relevance. Mobile and distributed sensor networks have the potential to revolutionize the way in which information is collected, fused and disseminated. It is closely related to distributed data fusion network \cite{Julier1997}, which attracts interest in many areas with its advantage of scalability, modularity and graceful degradation of performance in case of failures.

There are a number of avenues for future work. Firstly, the path planning with information entropy measure should be used with different problem set up. A good localization along the path is also important in surveying and environment sensing. Secondly, there is opportunity to extend the planning by introducing the presence of peer vehicles. For example, the goal can be changed to optimize the localization performance of the whole team or a particular vehicle, or to maximize the communication quality for information exchange. There could be some spatial correlation in the measurements among the team, for example, due to the tidy changes. The cooperative AUVs can be used for bathymetry map building. With knowledge on partial bathymetry map, the paths can be planned adaptively, and a fuller map can be developed from there.